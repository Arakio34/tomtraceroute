\documentclass[5pt]{article}
\usepackage{array}
\usepackage{listings}
\usepackage{graphicx}
\usepackage{color}
\usepackage{hyperref}
\usepackage{blindtext}
\title{Rapport M3102 TD/TP} 
\author{TUELEAU Tom}
\date{Janvier 2022}


\begin{document}
    \maketitle    
    \tableofcontents
    \newpage
    \section{Presentation}
    Pour la redaction de ce rapport je me suis servis des etapes donnees sur moodle afin de rediger ce document. Tout les sources utiliser seront citer tout au long du document. Tout le long des TP et TD j'ai stocker mon travail sur un repos github. Le lien est le suivant \url{https://github.com/Arakio34/tomtraceroute}.
    
    %Liste des rfc lue

    \section{Description des rfc}
    Lors de cette partie nous verrons les RFC que j'ai trouver pertinente pour d'écrire le fonctionnement des réseaux. En premier lieux les rfc que j'ai décider de vous presenter sont les suivante 
    \begin{center}
        \begin{tabular}{|l|l|}
            \hline
            Numéros de rfc & Titre \\
            \hline
            rfc793 & TCP \\
            \hline
            rfc768 & UDP \\
            \hline
            rfc791 & IP \\
            \hline
            rfc3232 & Numéros de ports \\
            \hline
            rfc1918 & Réseaux privés \\
            \hline
        \end{tabular}
    \end{center}
    
    %Liste des protcol
    \section{Definir les protocoles les plus utilises sur internet}
    Une fois les recherches effectuer sur les RFC faite, nous avions a lister les protocoles les plus utiliser d'internet. Les protocole regisent internet sont majoriterment definie par les rfc vue precedement. Nous avons donc UDP et TCP, deux protocole de la couche transport qui  
    
    %Description de traceroute
    \section{Traceroute}
    Traceroute est un programme qui permet de tracer la route vers une URL/IP. Ce programme peut prendre plusieurs options, notament afin de preciser le protocole a utiliser pour tracer la route. Ces option sont celle qui nous interessent afin de compremdre le fonctionnement de traceroute.


    %Description du script creer
    \section{Script}
    
    %Annalyse des anomalies trouver
    \section{Anomalies}


\end{document} 


%
#!/bin/bash

echo > dote.dot

function Help () {
	echo "##########################"
	echo "      Option d'aide"
	echo "##########################"
	echo "Utilisation :"
	echo "  sudo ./traceroute [OPTION] ..."
	echo ""
	echo "Options :"
	echo "  -f :                  Permet de de passer un fichier en option."
	echo "  -h :                  Affiche la page d'aide."
	echo "  -u :                  Cette option permet de specifier une URL/IP."
	echo "  -a :                  Cette option permet d'utiliser la liste des URL/IP de base."
	echo ""
	echo "Exemples :"
	echo "  sudo ./traceroute -f url.ls"
	echo "  sudo ./traceroute -u www.perdue.com"
	echo ""
	echo "url.ls :"
	echo "  www.google.com"
	echo "  www.perdue.com"
	echo "  www.iutbeziers.fr"
	echo "  www.youtube.com"
	echo "##########################"
}

function Trsolo () {
    protocole=("-I") # "-U" "-T" "-T-p80" "-T-p22" "-T-p20")
	declare -A shapePRO=( ['-U']="normal" ['-I']="diamond" ['-T']="vee" ['-D']="crow" ['-T-p80']="tee" ['-T-p22']="box" ['-T-p20']="dot" )
	color=( "red" "blue" "purple" "green" "brown" "coral" "darkorange" "gray" "gold" "pink" "cyan" "silver" "tomato" "slateblue" "webmaroon" "skyblue")
	echo > dote.dot
	urlia=$@
    
    # Debut de la carte 
    echo "digraph A {" >>dote.dot

    # Creation de la legende. 
    for pro in ${protocole[@]}
	do
		shape=${shapePRO[$pro]}
		echo "\"PROTOCOLE\"->\"prot = $pro\"->\"$shape\"[arrowhead=$shape]" >> dote.dot
	done
    
    y=0
	for url in ${urlia[@]}
	do
		echo " \"URL\" -> \"url = $url\"->\"${color[$y]}\"[color=${color[$y]}]" >> dote.dot
		y=$y+1
	done

    # Debuts de la route.
    i=0
	for url in ${urlia[@]}
	do	
		echo -e "Traceroute sur \e[31m$url \e[39m..."
		for pro in ${protocole[@]}
		do
			echo "$pro"
			echo > $url$pro 
			prob=$(echo $pro | sed "s/\-/ \-/g") 
			traceroute -A $prob $url >> $url$pro
			sed -i '2d' $url$pro                
			cat $url$pro | grep -o -E "(\([[:digit:]]{1,3}\.[[:digit:]]{1,3}\.[[:digit:]]{1,3}.[[:digit:]]{1,3})|(AS[[:digit:]]{1,})|(\* \* \*)" > p$url$pro 
			rm $url$pro
			element=$(cat p$url$pro |sed "s/(/ /g"|sed -r -e':a;N;$!ba;s/((\* \* \*\n))//g'| sed ':a;N;$!ba;s/\nA/ A/g' | sed ':a;N;$!ba;s/\n/\"\-\>\"/g')
			shape=${shapePRO[$pro]}
			bienfini=$(tail -n1 p$url$pro)
			if [ "$bienfini" = "* * *" ]
			then
				echo "\"$element\"[arrowhead=$shape, color=${color[$i]}]" >> dote.dot 
			else
				echo "\"$element\"->\"$url\"[arrowhead=$shape, color=${color[$i]}]" >> dote.dot
			fi
			rm p$url$pro
		done
		i=$i+1
	done

	#Fin du digraph
	echo "}" >> dote.dot
    mv dote.dot dote1.dot
    
    #On retire les bulles vides.	
    cat dote1.dot | sed -r "s/(\"\"\->(.){1,})//g" > dote.dot
	dot -Tpng dote.dot > dote.png
	rm dote1.dot
	exit
}


function Trfichier () {
	file=$1
	a=$(cat $file)
	urlfichier=$(echo $a)
	Trsolo $urlfichier
	exit
}

function Predef () {
    tab_URL=("www.umontpellier.fr" "www.iutbeziers.fr" "www.google.com" "ac-versailles.fr" "ac-montpellier.fr" "www.idf.iut.fr" "8.8.8.8" "1.1.1.1" "cisco.fr" "cisco.com" "stormshield.eu" "www.ac-paris.fr" "ac-toulouse.fr" "ac-lyon.fr" "ac-clermont.fr" "ac-bordeaux.fr")
    Trsolo ${tab_URL[@]}
}


while getopts "ahu:f:" option; do
    case "$option" in
        h) # Affiche l'aide.
            echo "Page d'aide."
            Help
            exit
            ;;
    	f) # Utilise un fichier pour les URL/IP.
            echo "Mode fichier." 
	        fileurl=$OPTARG
	        Trfichier $fileurl
	        exit
            ;;
        u) # Permet de donnes une URL/IP en arguments.
            echo "Mode solo URL." 
            url=$OPTARG
            Trsolo $url
            exit
            ;;
        a) # Utilise les URL et IP predefinie.
            echo "Mode predefinie." 
            Predef
    	    exit
            ;;
        *)
            Help
            exit
            ;;
    esac
done
Help
