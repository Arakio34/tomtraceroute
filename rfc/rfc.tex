\documentclass[5pt]{article}
\usepackage{array}
\usepackage{listings}
\usepackage{multibib}
\usepackage{graphicx}
\usepackage{color}
\usepackage{hyperref}
\usepackage{blindtext}
\title{Rapport M3102 TD/TP} 
\author{TUELEAU Tom}
\date{Janvier 2022}


\begin{document}
    \maketitle    
    \tableofcontents
    \newpage
    \section{Presentation}
    Le travail a effectuer dans cette partie est de trouver et comprendre les principal RFC qui regise internet. Pour la redaction de ce rapport je me suis servis des etapes donnees sur moodle afin de rediger ce document. Tout les sources utiliser ce trouve dans la section sitographie.
    
    %Liste des rfc lue

    \section{Description des rfc}
    Lors de cette partie nous verrons les RFC que j'ai trouver pertinente pour d'ecrire le fonctionnement des reseaux. 
    \subsection{rfc768 (UDP)}
    \subsection{rfc793 (TCP)}
    \subsection{rfc792 (ICMP)}
    \subsection{rfc791 (IP)}
    \subsection{rfc1105 (BGP)}
    
    %Liste des protcol
    \section{Definir les protocoles les plus utilises sur internet}
    Une fois les recherches effectuer sur les RFC faite, nous avions a lister les protocoles les plus utiliser d'internet. 
    
    %Description du paquet traceroute
    \section{Traceroute}
    Traceroute est un programme qui permet de tracer la route vers une URL/IP. Ce programme peut prendre plusieurs options, notament afin de preciser le protocole a utiliser pour tracer la route. Ces option sont celle qui nous interessent afin de compremdre le fonctionnement de traceroute.


    %Description du script creer
    \section{Script}
    
    %Annalyse des anomalies trouver
    \section{Anomalies}

    %Sitographie
    \section{Sitographie}
    \blibliographystyle{plain}
    \blibliography{rfc}

\end{document} 
